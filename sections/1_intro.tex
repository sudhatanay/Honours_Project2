\begin{multicols}{2}
\tableofcontents
\section{Introduction}
Wong Kar Wai is widely regarded as one of the pioneering filmmakers of the Hong Kong new wave. His films have gained international fame and possess unique stylistic elements such as vivid cinematography along with atmospheric music.

Additionally, Wong Kar Wai's films include non linear narratives or narratives that often depict a slice on ones life, allowing viewers to follow them for a particular time period or phase of the characters lives. His films often focus n the interactions between the various characters in his stories and the way they navigate the spaces they find themselves in.

\subsection{Filmography}
The films covered as part of this analysis are selectively chosen from his filmography between 1990 and 2000 which includes the films:

\begin{itemize}
    \item Days of Being Wild (1990)
    \item Chungking Express (1994)
    \item Fallen Angels (1995)
    \item In the Mood for Love (2000)
\end{itemize}

These films will be analyzed individually. Since each of these films are not heavily related to the others in terms of plot or story (even if several of these films are considered to be part of an informal trilogy such as \emph{Days of being wild, Chungking express and 2046} due to the presence of recurring themes or characters), a final section will be included in this paper that aims to summarize or describe Wong Kar Wai's cinematography using common elements from the films chosen. 

\end{multicols}