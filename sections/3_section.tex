{\color{gray}\hrule}
\begin{center}
\section{Common Themes}
\textbf{Various commonalities between Wong Kar Wai's filmography}
\end{center}
{\color{gray}\hrule}
\begin{multicols}{2}
\subsection{Unrequited Love}
A common theme across most of Wong Kar Wai's films is love. More specifically, it is unrequited love, where relationships do not materialize due to either a lack of interest from one party (such as in the case of Ho Chi-Mo and Charlie in \emph{Fallen Angels}, or circumstances causing pairs to drift apart (such as in the case of Su Li-Zhen and Chow Mo-Wan in \emph{In the mood for love}).

Much like in real life, putting two people together wouldn't necessarily guarantee a romantic relationship forming between them. Since characters of different stories often pass each other in Wong Kar Wai's films, it effectively serves as a deterrent since it implies that people and moments come and go. Even when two people spend time with each other more often, personal conflicts and ambitions often come in the way, preventing love from prevailing above all else. Thus Wong Kar Wai maintains a sense of realism when it comes to love and forming relationships and it is common to have unrequited love as a center theme across all of his stories and films.

\subsection{Usage of Narration /Narrative Monologues}
Throughout Wong Kar Wai's films, the protagonists often use internal monologues to convey their thoughts and emotions, allowing the viewer to effectively emphasise or connect with the main characters by understanding their actions or rationale behind said actions. This is present in nearly every film and serves and an effective tool to educate the audience about the characters themselves while visuals establish the space they are in These films have little to no political messages or any agenda but instead highlight and cover the independence of human beauty

\subsection{Usage of Space}
Most of Wong Kar Wai's filmography is set in urban spaces and follow the lives of characters that live in these spaces. Through the usage of cinematography (Wong Kar Wai often uses Christopher Doyle as his cinematographer of choice in several of his films) , Wong Kar Wai is effectively able to allow for small spaces to highlight his characters while at the same time utilizing large spaces to convey a sense of alienation and boredom/monotony. Techniques such as sped up shots with low frame rates (as used in chase sequences of \emph{Chungking Express}) give a sense of thrill and help immerse the viewer in the dense urban landscape that most of the stories are set in. Characters from other stories are often seen in shots, not as Easter eggs but as a simple consequence of the protagonists sharing a common space, being the city of Hong Kong.

\subsection{Universality of story}
While Wong Kar Wai mostly situates his stories around Hong Kong and urban Asia (with the exception of \emph{Happy Together}, the stories aren't inherently 'Hong Kong-like' in nature. These stories can be applied universally and resonate with audiences across the world. This is because Wong Kar Wai emphasises on individuality and character interactions over story, setting and ending. What could be a love story about a secret affair in 1960's Hong Kong could well apply to another location across the world. The usage of internal monologues also helps drive forth the importance of characters over the larger story. Often, there is no larger story present but instead follows these characters through particular moments of their lives in order to experience their way of living better.

\subsection{Usage of recurring actors and characters}
Wong Kar Wai's filmography often use the same actors across films. For example Maggie Cheung and Tony Leung appear in many films across Wong Kar Wai's filmography. Yet however, they play different roles across these films, cementing Wong Kar Wai's faith in these actors. In some cases, actors are often speculated to play similar yet distinct roles across different films (such as Wong Kar Wai's informal trilogy) which adds to the references and themes that are shared between many of his films.

\subsection{Ethereal settings}
In films such as \emph{Chungking Express, Fallen angels} and \emph{In the mood for love}, Wong Kar Wai utilizes cinematography and editing techniques to make Hong Kong seem dream-like, almost ethereal in nature. While the stories and the protagonists are grounded, the urban setting that is often highlighted by vivid cinematography add visual flair to these films. Even with tonally darker films such as \emph{Fallen Angels}, Wong Kar Wai is effectively able to maximize the visual aesthetics of his set locations evoking a sense of nostalgia and wonder in viewers.

\end{multicols}


