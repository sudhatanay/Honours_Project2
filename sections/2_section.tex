\setlength{\headheight}{15pt}
{\color{gray}\hrule}
\begin{center}
\section{Individual analysis}
% \textbf{You can add small descriptions of what the current sections describe}
\end{center}
{\color{gray}\hrule}
\begin{multicols}{2}
\subsection{Days of Being Wild (1990)}
\emph{Days of Being Wild} is a drama film that is one of Wong Kar Wai's earliest films. The film follows Yuddy (played by Leslie Cheung), the protagonist, a disillusioned young adult whose personality can be best described as a carefree playboy. He convinces Li-Zhen (played by Maggie Cheung) to be in a relationship with her. However, as time progresses, Li-Zhen wants to marry Yuddy, which he refuses and ends the relationship, leaving her heartbroken. Soon after, Yuddy dates Mimi (played by Carina Lau), a cabaret dancer. Yuddy's friend Zeb (played by Jacky Cheung) also harbours feelings for Mimi; however, Mimi is engrossed with Yuddy, leaving little to no regard for Zeb’s feelings.

Yuddy often utters the quote, \emph{“I've heard that there's a kind of bird without legs that can only fly and fly and sleep in the wind when it is tired. The bird only lands once in its life... that's when it dies.”} which he uses to describe himself as a man who is living life the way he wants to, with little to no intention of changing or stopping until his final moments. Yuddy also has a rough relationship with his adoptive mother, who was a former prostitute who now lives her life as a well-off escort to younger men, something with Yuddy disapproves of. Their relationship eventually breaks down as Yuddy is told the truth about his mother, resulting in Yuddy leaving for the Philippines in search of her, leaving Mimi and Zeb behind.

Meanwhile, Li-Zhen begins a friendship with a policeman named Tide, who offers comfort and solace to Li-Zhen in the form of a friend who listens and keeps her company. Tide reveals to Li-Zhen that he wanted to become a sailor but instead became a policeman to take care of his sick mother. After his mother passes away, Tide leaves to become a sailor. Mimi is distraught after finding out that Yuddy has left her behind for good and resolves herself to follow him. Zeb, as a final act of love for Mimi, decides to sell his car (that he obtained as a farewell gift from Yuddy) to fund her trip to find Yuddy.

Yuddy eventually finds himself in the Philippines. He tries to visit and meet his mother but instead is turned away, resulting in him becoming an aimless drunk wandering the streets of the Philippines. Tide, who is now a sailor, stumbles upon a passed-out Yuddy on a street in the Philippines and takes care of him by allowing Yuddy to stay in his quarters. The two men bond over their life stories up until that point, and Yuddy decides to leave the Philippines by procuring an illegal passport from some goons at a train station. 

The deal goes awry, and Yuddy ends up stabbing a man, after which Tide and Yuddy escape on a train. Tide then asks Yuddy if he remembers his time with Li-Zhen, to which Yuddy admits he does but insists that Li-Zhen not know about this. Yuddy is eventually shot dead when Tide momentarily leaves him with the goons he had escaped from earlier. Yuddy now interprets his quote differently, saying, \emph{“I used to think there was a kind of bird that, once born, would keep flying until death. The fact is that the bird hasn't gone anywhere. It was dead from the beginning.”}

Much like Wong Kar Wai’s future work, the film does not have a definite ending but instead simply stops following the protagonist. The world moves on without him, as do the characters he has interacted with. Yuddy’s story evokes melancholic feelings inside viewers while at the same time evoking responses of indifference at the end in a strange juxtaposition. A man who seemed to have everything, including a relationship, a caring foster mother and a close friend, had abandoned them all in pursuit of his real mother, which eventually resulted in him throwing his life away, amounting to nothing in the end. \emph{‘Days of being wild’} thus captures the moments of Yuddy’s life as he recollects his emotions and feelings through his monologue while also effectively showcasing a tragedy unfolding in front of our eyes.



\subsection{Chungking Express (1994)}
\emph{Chungking Express} is one of Wong Kar Wai’s most lighthearted films. It features two separate story-lines, with both story-lines featuring cops as the main protagonists. The first one’s protagonist is He Qiwu (played by Takeshi Kaneshiro), who plays a cop that is lovesick after his previous relationship had fallen apart. He finds himself calling old contacts in hopes of either reconnecting with his old love or trying to establish a contact in hopes of his feelings blossoming into something new. We also follow a woman in a blonde wig (played by Brigitte Lin) who appears to be associated with the drug underworld as we see her micromanaging drug mules as part of her assignment. However, the mules she had been preparing go missing, resulting in her realising she had been set up. 

These two end up encountering each other in a bar, each of them being exhausted from their ordeals, one being emotional (He Qiwu’s feelings of loneliness) and the other being more physical (The lady’s escape from the underworld goons). He Qiwu, who decides it is time to let go of his old relationship, decides to make a move at the lady, who appears uninterested in his advances. He Qiwu shares his story with her as she passes out in the bar from exhaustion. He Qiwu takes her to a hotel room and keeps her company throughout the night as he occupies himself with watching films as she sleeps and regains her energy. The following morning, the lady leaves the room and ties up loose ends by shooting the person responsible for setting her up. He Qiwu continues on his own way after waking up and receives birthday wishes on his pager, presumably from the lady. The two continue with their own lives, reinvigorated with a fresh outlook by each other's brief company and actions, ending the first story.

The second story features an amusing yet heartwarming story about a cop whose number is 663 (Played by Tony Leung) and Faye (Played by Faye Wong), a worker at a snack food store. Cop 663 also deals with heartbreak in the form of separation from a flight attendant. While it appears to be a short affair for the attendant, Cop 663 catches feelings for her and is saddened when he realises she has left him. Cop 663 visits the snack store often, and Faye falls for him. Cop 663’s ex-partner, who is a flight attendant, comes to the snack shop to return the keys to his house and hands over the keys to the shop's owner. Faye uses these keys to go to the cop's house when he isn't home and uses the time to clean up and organise his house, which was poorly maintained due to a lack of consideration from Cop 663. This goes on for a while until Cop 663 catches Faye in the act of cleaning his house. This results in Cop 663 asking Faye out on a date to which she initially agrees to come. However, she ends up not coming and leaving for California to chase her dreams. A year later, Faye is now a flight attendant and finds out that Cop 663 has bought the snack bar. While they do not end up in a relationship, the film ends with Cop 663 telling Faye he will go wherever she takes him.

This film offers a glimpse into the lives of the four protagonists spread over two different stories. These stories themselves have no intrinsic meaning but show the viewer how people come and drift apart by using heartbreak and relationships as the catalyst for events and occurrences. None of the supposed ‘couples’ ends up together, yet it still offers a sort of meaningful ending to each character as they learn to find their place in the world and the meaning of their lives. This is well aided by the narration that the two cops have in their stories as we follow their emotions and feelings of being lost and sad. The film also has several moments of comical relief in the form of amusing character interactions instead of the conventional slapstick comedy often seen in films today. Overall, \emph{Chungking Express} doesn't tell a story; instead, it simply puts viewers into the lives of the four characters and allows them to view a part of their life. The stories are more lighthearted that the ones in Wong Kar Wai's other films thus allowing \emph{Chungking Express} to revel in its innocence.

\subsection{Fallen Angels (1995)}
Wong Kar Wai initially intended for \emph{Fallen Angels} to be the third story following the two stories shown in \emph{Chungking Express}. However, due to film length constraints, \emph{Fallen Angels} ended up becoming its own film. While the film utilises techniques and directing similar to Chungking Express's, it is visually and tonally different as the story-line scenes are exclusively shot at night time. Like \emph{Chungking Express}, this film features two story-lines that function as independent stories but allow for the characters to mingle between the stories (such as the connection between the hit-man agent (played by Michelle Reis) and Ho Chi-Mo (played by Takeshi Kaneshiro)) allowing for a richer narrative.

The first story centres around a hit-man (played by Leon Lai) and his partner/agent. Their job is to complete contracts while maintaining little to no direct contact with each other. They communicate through faxes and letters. While the hit-man goes about his job and his assistant cleans up after him, his assistant is infatuated with her partner, as seen by her actions and feelings conveyed. She desperately tries to connect with him by foraging through the trash he leaves behind or pleasuring herself in hopes of being closer to her partner. From her viewpoint, he is mysterious and leaves little to decipher, yet none the less still finds herself in love with him. The hit-man, through his monologues, comments about how Partners should never be emotionally involved with each other. He is a lonely man who lives a solitary life and wonders if he has free will. As the hit-man carries out his job, he realises that he doesn't want to continue being a hit-man and conveys to his assistant through code that she must forget him.

He then goes on to start a relationship with a prostitute named “Blondie”, which the hit-man’s assistant finds out due to a chance encounter between the two women passing by each other, resulting in the assistant being cold towards the hit-man. Despite her disappointment, the hit-man ends his relationship with Blondie and begins working on his final job, one given to him by his assistant. However, this job ends up being a setup as the hit-man ends up losing his life. In his dying moments, however, the hit-man narrates that he can finally achieve free will and die.

The second story features Ho Chi-Mo, who is a mute man with a cheery yet chaotic disposition. He is a convict who has escaped prison, and the hit-man’s agent (from the earlier story) had even helped him shake off the cops once. At night, he breaks into places and masquerades as a business owner, selling his products to customers by force, which is played off as comedic. He runs into a woman frequently at night, and her name is revealed to be Charlie (played by Charlie Yeung). She is a possessive woman who is going through a breakup and uses Ho Chi-Mo as emotional support. The two of them bond by searching for her ex-partner, and Ho Chi-Mo ends up falling for her. Eventually, she ends up disappearing one day from his life. 

Charlie brings about a change in Ho Chi-Mo, who begins to take his life seriously and starts working in a job along with spending time with his father and capturing videos of themselves using a video camera. Unfortunately, his father passes away, and he slips into his old habits. He even re-encounters Charlie; however, she appears stable now, works as an attendant, and has a new partner. The film ends with the two stories resolving each other, as we see Ho Chi-Mo meet the hit-man agent, who offers her a ride. While they claim that no relationship is possible between them, they find comfort in each other briefly as they drive through the streets of Hong Kong on a bike together. The film contains much darker story-lines and undertones than \emph{Chungking Express} while still having the same sense of narrative style to it, allowing it to be a worthy successor to the critically acclaimed \emph{Chungking Express}.

\subsection{In the Mood for Love (2000)}
\emph{In the Mood for Love} is arguably one of Wong Kar Wai’s best films. The story centres around two protagonists named Su Li-Zhen (Played by Maggie Cheung) and Chow Mo-Wan (Played by Tony Leung), who form a bond together after discovering that their respective spouses are having an affair. The story is set in British Hong Kong in the 1960s Chow Mo-Wan, who is a journalist, recruits Su Li-Zhen to help him write a new serial, and they grow closer to each other as they discover that their spouses are having an affair.

They try to role-play as their spouses and practice confrontations with each other, which helps them bond even more. In order to not raise the suspicion of their neighbours about a possible affair between them, Chow Mo-Wan rents a hotel room in which the two of them work together. As they develop feelings for each other, they come to realise that they are no better than their spouses, who are already in an affair and struggle to be together openly. Chow then decides to move to Singapore for a new job and asks Su Li-Zen to accompany him. However, circumstances prevented this from happening.

Each of them tries to reach out to the other in the following years but decides not to initiate contact, as they are unsure about the other person's condition. Chow mentions a story about how in older times if one had a secret, they would go atop a mountain, whisper a secret in a tree hollow and cover it with mud to ensure that the secret remained hidden and stored away forever. At the end of the film, we see Chow visit the Angkor Wat and whisper something in a tree hollow before covering the hollow with mud, implying that he has a secret, likely regarding Su Li-Zhen.

The film is highly sensual in nature, even if the end result doesn't pair the two protagonists together. The usage of close spaces, such as the apartment complex they live in, and the hustle and bustle of the neighbours, such as Mrs Suen, give the impression that this is a budding romance that isn't conventional in nature, thus requiring secrecy and delicate treading. The film is also heavily atmospheric in nature. It conveys the attraction between the main protagonists through visual storytelling as well as the repeated use of a song, indicating the passage of time and the monotony of exchanges between Su Li-Zhen and Chow Mo-Wan. Wong Kar Wai intended for the setting and visuals of this film to evoke nostalgia for 1960s Hong Kong when immigrants from Shanghai began to form pockets and settle in Hong Kong. This film and the film before it, “Happy Together”, saw Wong Kar Wai take on projects at an international level and not just restrict set locations in Hong Kong exclusively, with set locations that are not exclusively situated in Hong Kong.

\end{multicols}